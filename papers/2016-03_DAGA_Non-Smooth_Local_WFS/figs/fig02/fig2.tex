%*****************************************************************************
% Copyright (c) 2016      Fiete Winter                                       *
%                         Institut fuer Nachrichtentechnik                   *
%                         Universitaet Rostock                               *
%                         Richard-Wagner-Strasse 31, 18119 Rostock, Germany  *
%                                                                            *
% This file is part of the supplementary material for Fiete Winter's         *
% scientific work and publications                                           *
%                                                                            *
% You can redistribute the material and/or modify it  under the terms of the *
% GNU  General  Public  License as published by the Free Software Foundation *
% , either version 3 of the License,  or (at your option) any later version. *
%                                                                            *
% This Material is distributed in the hope that it will be useful, but       *
% WITHOUT ANY WARRANTY; without even the implied warranty of MERCHANTABILITY *
% or FITNESS FOR A PARTICULAR PURPOSE.                                       *
% See the GNU General Public License for more details.                       *
%                                                                            *
% You should  have received a copy of the GNU General Public License along   *
% with this program. If not, see <http://www.gnu.org/licenses/>.             *
%                                                                            *
% http://github.com/fietew/publications           fiete.winter@uni-rostock.de*
%*****************************************************************************

\tikzstyle{loudspeaker} = [basic loudspeaker, draw, fill=white
, minimum height=1.0mm 	    % height of the rectangle ("driver") part 
, minimum width=0.5mm		    % width of the rectangle
, inner sep=0pt
, relative cone width=1	% height of trapezoid ("cone") part, relative to 
%minimum width (default: 1)
, relative cone height=2	    % width of trapezoid, relative to minimum height 
%(default: 2)
]

\tikzstyle{active} = [fill=activecolor]
\tikzstyle{focus} = [draw, fill=white, circle, minimum size=2mm, inner sep=0pt]
\tikzstyle{point} = [draw, fill=black, circle, minimum size=1.0mm, inner 
sep=0pt]

\tikzstyle{corner} = [opacity=0.5]
\tikzstyle{corner1} = [corner, color=blue]
\tikzstyle{corner2} = [corner, color=red]
\tikzstyle{corner3} = [corner, color=teal]
\tikzstyle{corner4} = [corner, color=mauve]

\tikzstyle{line1} = [color=blue]
\tikzstyle{line2} = [color=red]
\tikzstyle{line3} = [color=teal]
\tikzstyle{line4} = [color=mauve]

\pgfdeclarelayer{background}
\pgfdeclarelayer{foreground}
\pgfsetlayers{background,main,foreground}

\begin{tikzpicture}

%%%%%%%%%%%%%%%%%%%%%%%%%%%%%%%%%%%%%%%%%%%%%%%%%%%%%%%%%%%%%%%%%%%%%%%%%%%%%%%%
%%% loudspeakers %%%
\pgfmathsetmacro{\N}{64}  % number of loudspeakers
\pgfmathsetmacro{\Nseg}{\N/4}  % number of loudspeaker per linear segment
\pgfmathsetmacro{\NsegM}{\Nseg-1}
\pgfmathsetmacro{\L}{3.0}  % length of linear segment
\pgfmathsetmacro{\Ldelta}{\L/(\NsegM)}  % distance between adjacent loudspeakers
% distance of linear segment to coordinates' origin
\pgfmathsetmacro{\R}{\L/2+1/sqrt(2)*\Ldelta}

%%% virtual point source %%
\pgfmathsetmacro{\Rs}{3.0}  % distance from origin
\pgfmathsetmacro{\phis}{50}  % azimuth
\pgfmathsetmacro{\Rlabel}{\R+0.2}  % position of position label
\pgfmathsetmacro{\Rphilabel}{0.5*\R}  % position of azimuth label
\pgfmathsetmacro{\Rtmp}{0.4*\R}

%%% misc. %%%
\pgfmathsetmacro{\Rcoord}{1.5*\R}  % length of coordinate axes
\pgfmathsetmacro{\Rout}{1.15*\R}  % outer distance of colored lines
\pgfmathsetmacro{\Rin}{1.10*\R}  % inner distance of colored lines
\pgfmathsetmacro{\Rcross}{0.6*\R}

%%%%%%%%%%%%%%%%%%%%%%%%%%%%%%%%%%%%%%%%%%%%%%%%%%%%%%%%%%%%%%%%%%%%%%%%%%%%%%%%
%%% loudspeakers domain %%
% listening area
\draw[thick, fill=focuscircle] (-\R,-\R) rectangle (\R,\R);
\foreach[count=\sdx] \phi in {0, 90, 180, 270} {
  
  \begin{pgfonlayer}{background}
    \coordinate (segment) at (\phi:\R);  % reference pos. of loudspeakers
    \coordinate (segmentshifted) at (\phi:1.15*\R);  % reference pos. of numbers
    % draw loudspeaker symbols
    \foreach \idx in {0,1,...,\NsegM}{
      \node (speaker\phi\idx) at 
      ($(segment)+(\phi+90:-\L/2+\idx*\Ldelta)$) 
      [loudspeaker,active,anchor=cone,rotate=\phi-180]{};
    };
  \end{pgfonlayer}

%%% crossfade illustration %%%
  \coordinate (corner) at ($(\phi:\R)+(\phi+90:\R)$);
  % rectangle for corner area
  \fill[corner\sdx] (corner) rectangle +($(\phi:\Rcross)+(\phi+90:\Rcross)$);
  
  % first crossfade illustration
  \coordinate (sinestart) at ($(corner) + (\phi:\Rcross)$);
  \coordinate (sinemid) at (\phi:\R+0.5*\Rcross);
  \coordinate (sineend) at ($(\phi:\R)+(\phi-90:\R)$);    
  \fill[corner\sdx, rotate=\phi+90] 
    (sineend) cos (sinemid) sin (sinestart) -- (corner);
  
  % second crossfade illustration
  \coordinate (sinestart) at ($(\phi+90:\R)+(\phi+90:\Rcross)+(\phi:\R)$);
  \coordinate (sinemid) at (\phi+90:\R+0.5*\Rcross);
  \coordinate (sineend) at ($(\phi+90:\R)+(\phi+180:\R)$);  
  \fill[corner\sdx, rotate=\phi+180]    
    (sineend) cos (sinemid) sin (sinestart) -- (corner);
};

%%% colored lines %%%
\draw[ultra thick, line1] (-\R,\Rin) -- (\Rin,\Rin) -- (\Rin,-\R);
\draw[ultra thick, line2] (\R,\Rout) -- (-\Rout,\Rout) -- (-\Rout,-\R);
\draw[ultra thick, line3] (-\Rin,\R) -- (-\Rin,-\Rin) -- (\R,-\Rin);
\draw[ultra thick, line4] (-\R,-\Rout) -- (\Rout,-\Rout) -- (\Rout,\R);

%%% virtual sound field %%%
\draw (0,0) -- (\phis:\Rs);
\node[point] at (\phis:\Rs){};
\node[right=3pt] at (\phis:\Rs){$\sfposs$};
\draw[-latex] (0:\Rphilabel) arc (0:\phis:\Rphilabel);
\node[right] at (0.5*\phis:\Rphilabel) {$\sfsphphi[subscript=s]$};

%%% coordinate axes %%%
\draw[thick, -latex] (-\Rcoord,0) -- (\Rcoord,0) node[at end, below] {$x$};
\draw[thick, -latex] (0,-\Rcoord) -- (0,\Rcoord) node[at end, left] {$y$};

%%% reference point %%%
\node[point] at (0,0){};
\node[below right] at (0,0){$\sfposref$};

\end{tikzpicture}